\documentclass[11pt,]{article}
\usepackage[margin=1in]{geometry}
\newcommand*{\authorfont}{\fontfamily{phv}\selectfont}
\usepackage[]{mathpazo}
\usepackage{abstract}
\renewcommand{\abstractname}{}    % clear the title
\renewcommand{\absnamepos}{empty} % originally center
\newcommand{\blankline}{\quad\pagebreak[2]}

\providecommand{\tightlist}{%
  \setlength{\itemsep}{0pt}\setlength{\parskip}{0pt}} 
\usepackage{longtable,booktabs}

\usepackage{parskip}
\usepackage{titlesec}
\titlespacing\section{0pt}{12pt plus 4pt minus 2pt}{6pt plus 2pt minus 2pt}
\titlespacing\subsection{0pt}{12pt plus 4pt minus 2pt}{6pt plus 2pt minus 2pt}

\titleformat*{\subsubsection}{\normalsize\itshape}

\usepackage{titling}
\setlength{\droptitle}{-.25cm}

%\setlength{\parindent}{0pt}
%\setlength{\parskip}{6pt plus 2pt minus 1pt}
%\setlength{\emergencystretch}{3em}  % prevent overfull lines 

\usepackage[T1]{fontenc}
\usepackage[utf8]{inputenc}

\usepackage{fancyhdr}
\pagestyle{fancy}
\usepackage{lastpage}
\renewcommand{\headrulewidth}{0.3pt}
\renewcommand{\footrulewidth}{0.0pt} 
\lhead{}
\chead{}
\rhead{\footnotesize PLCP: Introduction to Comparative Political Economy -- Fall 2018}
\lfoot{}
\cfoot{\small \thepage/\pageref*{LastPage}}
\rfoot{}

\fancypagestyle{firststyle}
{
\renewcommand{\headrulewidth}{0pt}%
   \fancyhf{}
   \fancyfoot[C]{\small \thepage/\pageref*{LastPage}}
}

%\def\labelitemi{--}
%\usepackage{enumitem}
%\setitemize[0]{leftmargin=25pt}
%\setenumerate[0]{leftmargin=25pt}




\makeatletter
\@ifpackageloaded{hyperref}{}{%
\ifxetex
  \usepackage[setpagesize=false, % page size defined by xetex
              unicode=false, % unicode breaks when used with xetex
              xetex]{hyperref}
\else
  \usepackage[unicode=true]{hyperref}
\fi
}
\@ifpackageloaded{color}{
    \PassOptionsToPackage{usenames,dvipsnames}{color}
}{%
    \usepackage[usenames,dvipsnames]{color}
}
\makeatother
\hypersetup{breaklinks=true,
            bookmarks=true,
            pdfauthor={ ()},
             pdfkeywords = {},  
            pdftitle={PLCP: Introduction to Comparative Political Economy},
            colorlinks=true,
            citecolor=blue,
            urlcolor=blue,
            linkcolor=magenta,
            pdfborder={0 0 0}}
\urlstyle{same}  % don't use monospace font for urls


\setcounter{secnumdepth}{0}





\usepackage{setspace}

\title{PLCP: Introduction to Comparative Political Economy}
\author{Robert Kubinec}
\date{Fall 2018}


\begin{document}  

		\maketitle
		
	
		\thispagestyle{firststyle}

%	\thispagestyle{empty}


	\noindent \begin{tabular*}{\textwidth}{ @{\extracolsep{\fill}} lr @{\extracolsep{\fill}}}


E-mail: \texttt{\href{mailto:rmk7xy@virginia.edu}{\nolinkurl{rmk7xy@virginia.edu}}} & Web: \href{http://www.robertkubinec.com}{\tt www.robertkubinec.com}\\
Office Hours: W 09:00-11:30 a.m.  &  Class Hours: XXX\\
Office: XXXX  & Class Room: \emph{XXX}\\
	&  \\
	\hline
	\end{tabular*}
	
\vspace{2mm}
	


\section{Course Description}\label{course-description}

Comparative political economy is the study of political institutions
that control economic activity and the ramifications that these
institutions have on political outcomes, including social movements,
elections and legislatures. The discipline has a very rich history,
dating as far back as Aristotle, but it received its current form during
the 18th and 19th centuries where we will begin our course overview.
Lately, research in comparative political economy has explored social
and political phenomena not considered to be heavily affected by
economic constraints, such as media coverage, sectarianism and the rebel
group recruitment. Through this broadening of political economy, the
field has become distilled into a set of principles, called rational
choice, that can be applied to a wide set of social institutions. In
this class, we will explore the traditional study of political-economic
outcomes, including the effects of income inequality, trade, corruption
and international financial institutions (IFIs), along with a summary of
trending research in election fraud, one-party states, and civil wars.

\section{Course Objectives}\label{course-objectives}

\begin{enumerate}
\def\labelenumi{\arabic{enumi}.}
\item
  Attain mastery of evaluation and critique of a wide range of readings
  in the discipline of political economy;
\item
  Employ the rational-choice approach in theoretical and applied
  settings through class discussion \& writing;
\item
  Engage in the scholarly process of discovery through an independent
  research project.
\end{enumerate}

\section{Required Texts}\label{required-texts}

The following two books have been purchased and are available at the
book store. I have also placed a copy of each on reserve at the library.
We will spend a week on each book as intensive deep-dive into the
arguments made by these authors and how they use political economy as a
methodology for research.

Arriola, Leonardo (2012).
\emph{Multiethnic Coalitions in Africa: Business Financing of Opposition Election Campaigns}.
Cambridge University Press.

Tilly, Charles (1992).
\emph{Coercion, Capital and European States, AD 990-1992}. Cambridge,
MA: Blackwell.

In addition for graduate students:

Acemoglu, Daron and James Robinson (2006).
\emph{Economic Origins of Dictatorship and Democracy}. Cambridge
University Press.

Bates, Robert H. (1981).
\emph{Markets and states in tropical Africa: the political basis of agricultural policies}.
University of California Pr.

Besley, Timothy (2007).
\emph{Principled Agents?: The Political Economy of Good Government}.
Oxford University Press.

Haber, Stephen, Armando Razo and Noel Maurer (2003).
\emph{The Politics of Property Rights: Political Instability, Credible Commitments, and Economic Growth in Mexico, 1876-1929}.
Cambridge University Press.

\section{Course Policies}\label{course-policies}

Please use the information below as a reference for how this class will
be conducted. I would ask that you review this information before
contacting me with any questions.

\subsection{Grading Policy}\label{grading-policy}

\begin{itemize}
\item
  \textbf{20\%} of your grade will be determined by a midterm during
  normal class hours.
\item
  \textbf{40\%} of your grade will come from a 15-20 page (graduate:
  25-30 page) paper that explores in further detail one of the research
  areas on the class syllabus. By the midterm students must have three
  hypotheses related to one of these research areas and a draft of
  available resources, which will constitute 5\% of the 40\%. I expect
  that students will use original data collection, whether quantitative
  or qualitative, to answer the questions posed.
\item
  \textbf{20\%} of your grade will be determined by your attendance and
  participation in class. I expect you to come having read the reading
  and prepared to pose at least one question or critique of the readings
  for that week.
\item
  \textbf{20\%} of your grade will come from three 2-page critical
  summaries of the reading that you must complete on three separate
  weeks of the class.
\end{itemize}

\subsection{Attendance Policy}\label{attendance-policy}

Your attendance and participation grade is divided into two parts: 10\%
comes from your presence in class, and 10\% from your participation
during discussion. I allow you to miss three classes during the semester
for any reason: you do not need to inform me ahead of time unless you
are missing class for an approved reason (i.e., an official university
event or activity) and would like the absence to be excused.

The second 10\% of the grade is qualitative in nature and based on my
assessment of your preparation for class and willingness to participate
in class discussions. The intent of this grade is not to reward
extroverts, but rather to emphasize that classroom learning depends on
the commitment of everyone in the class, students included. By asking
questions that come from a critical evaluation of the reading, you
provide the rest of the class with additional learning through the
ability to consider your perspective. My rule of thumb is for you come
to class with at least one question based on the reading that you would
like to ask.

\subsection{E-mail Policy}\label{e-mail-policy}

I welcome emails from students regarding class policies, assignments \&
readings. In general I respond to all emails within 24 hours; however,
it may take me longer to do so on the weekends. In addition, it is
unlikely that I will be able to respond very quickly before deadlines
such as exams and papers, so please provide at least one day for a
response if you need an issue clarified about an assigment.

I am open to taking questions about the readings and class discussions
via email, but I prefer to have substantive discussions in person at my
office hours.

\subsection{Make-Up Exam Policy}\label{make-up-exam-policy}

If an event outside of your control causes you to miss or reschedule an
exam, please email me as soon as possible so alternative arrangements
can be made. Please note that as a rule I do not reschedule examinations
for student vacations.

\subsection{Academic Dishonesty
Policy}\label{academic-dishonesty-policy}

Plagiarism undermines the very core of the mission of this class, which
is for each student to grow as an emerging scholar of political-economic
research. Without completing all the assignments yourself, you will
never be able to achieve objectives. Even worse, plagiarism reduces the
trust in each other's work that is a necessary component of the
scientific enterprise.

Plagiarism includes copying in whole or part from previously published
academic works without proper citation, but it also covers passing off
someone else's work as your own regardless of whether or not it has been
published. Hiring someone to write an essay for you is just as much
plagiarism as if you copied an existing article for your work.
Furthermore, it should go without saying that copying another student's
work in this class (with or without their permission) violates academic
integrity. For full details of what plagiarism means in a given
situation, refer to the university policy available from the registrar,
which I will use when needing to make determination about plagiarism.

If I determine that you have willfully plagiarized an assignment, you
will receive an automatic zero on the assignment and you may fail the
class or receive further disciplinary action per university policy.

\subsection{Diversity within the
Classroom}\label{diversity-within-the-classroom}

This class will explore issues that may be contentious. I expect that
all students treat each other with respect. This means that all
arguments in the class should be based on factual assertions as opposed
to demeaning insults. Furthermore, this class is designed to bring in
diverse viewpoints within the class into discussion, whether that
involve ethnic heritage, religious perspectives, political ideologies,
or racial categories. I want students to learn to see from each other's
points of view even if they disagree with what each other say, and to
learn to accept each other as fellow scholars. Every person in this
class will have an equal chance to speak and share their opinion with
the understanding that they must give each other the same respect and
understanding. Finally, I will not tolerate the denigration of anyone in
the class because of their adopted or prescribed social, religious,
political, ethnic, racial, gender-based or sexual identities.

\subsection{Additional Learning Needs}\label{additional-learning-needs}

Each of us has a different learning style, and I will do my best to
accomodate diverse learning needs in the class. If you need any kind of
accomodation, please come talk to me as soon as you can so we can
arrange a style of learning that works for you. I also refer you to the
university's Learning Needs Center for more information on resources
that you can use to help you get the most out of this class.

\newpage

\section{Class Schedule}\label{class-schedule}

I expect students to have read the assigned readings before class. This
does not mean just skimming reading, but engaging critically with the
scholarship. In particular, look for passages that you disagree with or
that seem unclear to you, as these are likely ones that could benefit
from further discussion in class. I recommend that, if at all possible,
students find a way to mark up the articles or books as they are read to
improve reading comprehension.

\subsection{Week 01, 08/27 - 08/31: Introduction \& Course
Expectations}\label{week-01-0827---0831-introduction-course-expectations}

\subsection{Week 02, 09/03 - 09/07: Traditions of Political Economy:
Marxists \&
Liberals}\label{week-02-0903---0907-traditions-of-political-economy-marxists-liberals}

Pages 1-50 from Smith, Adam (1776).
\emph{An Inquiry into the Nature and Causes of the Wealth of Nations}.
London: Methuen \& Co.

``Wage Labour and Capital'' and ``The Eighteenth Brumaire of Louis
Bonaparte'' from Marx, Karl (1978). \emph{The Marx-Engels Reader}. Ed.
by Robert C. Tucker. Norton.

Centeno, Miguel A. and Joseph N. Cohen (2012). ``The Arc of
Neoliberalism''. In: \emph{Annual Review of Sociology} 38, pp.~317-340.

North, Douglass C. (1991). ``Institutions''. In:
\emph{Journal of Economic Perspectives} 5.1, pp.~97-112.

Schwartz, Herman (2007). ``Dependency or Institutions? Economic
Geography, Causal Mechanisms, and Logic in the Understanding of
Development''. In:
\emph{Studies in Comparative International Development} 42.1-2,
pp.~115-135. DOI: 10.1007/s12116-007-9000-x.
\url{http://dx.doi.org/10.1007/s12116-007-9000-x}.

For graduate students:

Polyani, Karl (1944). \emph{The Great Transformation}. Beacon Press.

Schumpeter, Joseph (1976). \emph{Capitalism, Socialism and Democracy}.
Routledge.

\subsection{Week 03, 09/10 - 09/14: Political Economy of Development I:
Overview}\label{week-03-0910---0914-political-economy-of-development-i-overview}

Pages 1-50 from North, Douglass C, John Joseph Wallis and Barry R.
Weingast (2009).
\emph{Violence and Social Orders: A Conceptual Framework for Interpreting Recorded Human History}.
Cambridge University Press.

Geddes, Barbara (1991). ``A Game Theoretic Model of Reform in Latin
American Democracies''. In:
\emph{The American Political Science Review}, pp.~371-392.

North, Douglass C. and Barry R. Weingast (1989). ``Constitutions and
commitment: the evolution of institutions governing public choice in
seventeenth-century England''. In:
\emph{The journal of economic history} 49.04, pp.~803-832.

Olson, Mancur (1993). ``Dictatorship, Democracy, and Development''. In:
\emph{The American Political Science Review} 87.3, pp.~567-576.

Graduate students:

Chapters 1-3 from Haber, Stephen, Armando Razo and Noel Maurer (2003).
\emph{The Politics of Property Rights: Political Instability, Credible Commitments, and Economic Growth in Mexico, 1876-1929}.
Cambridge University Press.

\subsection{Week 03, 09/10 - 09/14: Political Economy of Development II:
Long-Run
Institutions}\label{week-03-0910---0914-political-economy-of-development-ii-long-run-institutions}

Chs. 1-3 from Waldner, David (1999).
\emph{State Building and Late Development}. Cornell University Press.

Chs. 1-2 from Evans, Peter (1995).
\emph{Embedded Autonomy: States and Industrial Transformation}.

Graduate students:

Acemoglu, Daron, Simon Johnson and James Robinson (2004).
\emph{Institutions as the Fundamental Cause of Long-Run Growth}. Working
Paper. National Bureau of Economic Research.
\url{http://www.nber.org/papers/w10481}.

Acemoglu, Daron and A Robinson (2001). ``The Colonial Origins of
Comparative Development: An Empirical Investigation''. In:
\emph{The American Economic Review} 91.5, pp.~1369--1401.

Blaydes, Lisa and Mark Andreas Kayser (2011). ``Counting Calories:
Democracy and Distribution in the Developing World''. In:
\emph{International Studies Quarterly} 55.4, pp.~887-908.

Geddes, Barbara (1991). ``A Game Theoretic Model of Reform in Latin
American Democracies''. In:
\emph{The American Political Science Review}, pp.~371-392.

Olson, Mancur (1993). ``Dictatorship, Democracy, and Development''. In:
\emph{The American Political Science Review} 87.3, pp.~567-576.

\subsection{Week 04, 09/17 - 09/21: Political Economy of Development
III: Ethnicity \&
Development}\label{week-04-0917---0921-political-economy-of-development-iii-ethnicity-development}

Chapters 1-2 from Herbst, Jeffrey (2000).
\emph{States and Power in Africa: Comparative Lessons in Authority and Control}.
Princeton University Press.

Acemoglu, Daron, Tristan Reed and James A Robinson (2013).
\emph{Chiefs: Elite Control of Civil Society and Economic Development in Sierra Leone}.
Working Paper. National Bureau of Economic Research.
\url{http://www.nber.org/papers/w18691}.

Habyarimana, James, Macartan Humphreys, Daniel N. Posner and Jeremy M.
Weinstein (2007). ``Why Does Ethnic Diversity Undermine Public Goods
Provision?'' In: \emph{American Political Science Review} 101.4,
pp.~709-725.

Graduate students:

Alesina, Alberto and Eliana La Ferrara (2004).
\emph{Ethnic Diversity and Economic Performance}. Working Paper.
National Bureau of Economic Research.

Michalopoulos, Stelios and Elias Papaioannou (2013). ``Pre-Colonial
Ethnic Institutions and Contemporary African Development''. In:
\emph{Econometrica} 81.1, pp.~113-152.

\subsection{Week 05, 09/24 - 09/28: Political Economy of Development IV:
Democracy \&
Development}\label{week-05-0924---0928-political-economy-of-development-iv-democracy-development}

Boix, Carles (2011). ``Democracy, Development, and the International
System''. In: \emph{American Political Science Review} 105.4,
pp.~809-828.

Boix, Carles and Susan Carol Stokes (2003). ``Endogenous
Democratization''. In: \emph{World politics} 55.4, pp.~517--549.

Limongi, Fernando and Adam Przeworski (1997). ``Modernization: Theories
and facts''. In: \emph{World politics} 49.2, pp.~155--183.

Lipset, Seymour M. (1959). ``Some Social Requisites of Democracy:
Economic Development and Political Legitimacy''. In:
\emph{The American Political Science Review} 53.1, pp.~69-105.

For graduate students:

Acemoglu, Daron and James A Robinson (2008). ``Persistence of Power,
Elites, and Institutions''. In: \emph{American Economic Review} 98.1,
pp.~267--93.

Acemoglu, Daron, James A Robinson, Simon Johnson and Pierre Yared
(2009). ``Reevaluating the Modernization Hypothesis''. In:
\emph{Journal of Monetary Economics} 56.8, pp.~1043-1058.

Albertus, Michael and Victor Menaldo (2014). ``Gaming Democracy: Elite
Dominane During Transition and the Prospects for Redistribution''. In:
\emph{British Journal of Political Sciene} 44.3, pp.~575-603.

Ansell, Ben W. and David J. Samuels (2010). ``Inequality and
Democratization: A Contractarian Approach''. In:
\emph{Comparative Political Studies} 43.12, pp.~1543--1574.

\subsection{Week 06, 10/01 - 10/05: Political Economy of Redistribution
I: Advanced Industrial
Economies}\label{week-06-1001---1005-political-economy-of-redistribution-i-advanced-industrial-economies}

Korpi, Walter (2006). ``Power Resources and Employer-Centered Approaches
in Explanations of Welfare States and Varieties of Capitalism:
Protagonists, Consenters, and Antagonists''. In: \emph{World Politics}
58.02, pp.~167-206. DOI: 10.1353/wp.2006.0026.
\url{http://dx.doi.org/10.1353/wp.2006.0026}.

Pierson, Paul (1996). ``The New Politics of the Welfare State''. In:
\emph{World Politics} 48.2, pp.~143-179.

Piketty, Thomas and Emmanuel Saez (2006).
\emph{The Evolution of Top Incomes: A Historical and International Perspective}.
Working Paper. National Bureau of Economic Research.
\url{http://www.nber.org/papers/w11955}.

Wallerstein, Michael (1999). ``Wage-Setting Institutions and Pay
Inequality in Advanced Industrial Societies''. In:
\emph{American Journal of Political Science} 43.3, pp.~649-680.

For graduate students:

Alesina, Alberto, Edward Glaeser and Bruce Sacerdote (2001).
\emph{Why Doesn't the United States Have a European-Style Welfare State?}
Papers on Economic Activity. Brookings Institution, pp.~187--254.

Boix, Carles (1999). ``Setting the Rules of the Game: The Choice of
Electoral Systems in Advanced Democracies''. In:
\emph{American Political Science Review}, pp.~609--624.

Iversen, Torben and David Soskice (2006). ``Electoral institutions and
the politics of coalitions: Why some democracies redistribute more than
others''. In: \emph{American Political Science Review} 100.02,
pp.~165--181.

\emph{Draft of paper hypotheses due for in-class critique}

\subsection{Week 07, 10/08 - 10/12: Political Economy of Redistribution
II: Redistribution in Late-Developing
Countries}\label{week-07-1008---1012-political-economy-of-redistribution-ii-redistribution-in-late-developing-countries}

Cammett, Melani and Lauren M. Maclean (2011). ``Introduction: The
Political Consequences of Non-State Social Welfare in the Global
South''. In: \emph{Studies in Comparative International Development}
46.1, pp.~1-21.

Mares, Isabela and Matthew E. Carnes (2009). ``Social Policy in
Developing Countries''. In: \emph{Annual Review of Political Science}
12, pp.~93-113.

Wibbels, Erik and John S. Ahlquist (2011). ``Development, Trade and
Social Insurance''. In: \emph{International Studies Quarterly} 2011.1,
pp.~125-149.

For graduate students:

Avelino, George, David S. Brown and Wendy Hunter (2005). ``The Effects
of Capital Mobility, Trade Openness, and Democracy on Social Spending in
Latin America, 1980-1999''. In:
\emph{American Journal of Political Science} 49.3, pp.~625-641.

Miller, Michael K. (2015). ``Electoral Authoritarianism and Human
Development''. In: \emph{Comparative Political Studies} 48.12,
pp.~1526-1562.

Weyland, Kurt (2005). ``Theories of Policy Diffusion: Lessons from Latin
American Pension Reform''. In: \emph{World Politics} 57.2, pp.~262-295.

\emph{Midterm Examination}

\subsection{Week 08, 10/15 - 10/19: Political Economy of Redistribution
III: Regime
Transitions}\label{week-08-1015---1019-political-economy-of-redistribution-iii-regime-transitions}

Deep Dive:

Arriola, Leonardo (2012).
\emph{Multiethnic Coalitions in Africa: Business Financing of Opposition Election Campaigns}.
Cambridge University Press.

For graduate students:

Acemoglu, Daron and James Robinson (2006).
\emph{Economic Origins of Dictatorship and Democracy}. Cambridge
University Press.

Acemoglu, Daron and James A Robinson (2008). ``Persistence of Power,
Elites, and Institutions''. In: \emph{American Economic Review} 98.1,
pp.~267--93.

Albertus, Michael and Victor Menaldo (2014). ``Gaming Democracy: Elite
Dominane During Transition and the Prospects for Redistribution''. In:
\emph{British Journal of Political Sciene} 44.3, pp.~575-603.

Ansell, Ben W. and David J. Samuels (2010). ``Inequality and
Democratization: A Contractarian Approach''. In:
\emph{Comparative Political Studies} 43.12, pp.~1543--1574.

\subsection{Week 09, 10/22 - 10/26: Political Economy of Trade I:
Globalization}\label{week-09-1022---1026-political-economy-of-trade-i-globalization}

Chapters 1-3 from Baghwati, Jagdish N. (2004).
\emph{In Defense of Globalization}. Oxford University Press.

Vogel, David (1997). ``Trading Up and Governing Across: Transnational
Governance and Environmental Protection''. In:
\emph{Journal of European Public Policy} 4.4, pp.~556-571.

Wood, A. and K. Jordan (2000). ``Why Does Zimbabwe Export Manufactures
and Uganda Not? Economics Meets History''. In:
\emph{The Journal of Development Studies} 37.2, pp.~91-116.

For graduate students:

Alt, James E, Jeffrey Frieden and Michael J. Gilligan (1996). ``The
Political Economy of International Trade: Enduring Puzzles and an Agenda
for Inquiry''. In: \emph{Comparative Political Studies} 29.6,
pp.~689-717.

Boix, Carles (2006). ``Between Protectionism and Compensation: The
Political Economy of Trade''. In:
\emph{Globalization and Egalitarian Redistribution}. Ed. by Pranab
Bardhan, Samuel Bowles and Michael Wallerstein. Princeton University and
Russel Sage Foundation.

Gourevitch, Peter (1978). ``The Second Image Reversed: The International
Sources of Domestic Politics''. In: \emph{International Organization}
32.4, pp.~881-912.

Mahler, Vincent A. (2004). ``Economic Globalization, Domestic Politics,
and Income Inequality in the Developed Countries''. In:
\emph{Comparative Political Studies} 37.9, pp.~1025-1053.

Milner, Helen V. and Keiko Kubota (2005). ``Why the Move to Free Trade?
Democracy and Trade Policy in the Developing Countries''. In:
\emph{International Organization} 59.1, pp.~107-143.

\subsection{Week 10, 10/29 - 11/02: Political Economy of Trade II:
Multinational
Firms}\label{week-10-1029---1102-political-economy-of-trade-ii-multinational-firms}

Kim, Song, Helen V. Milner, Thomas Bernauer, Gabriele Spilker, Iain
Osgood and Dustin Tingley (2017). ``Firms' Prefernces over
Multidimensional Trade Policies: Global Production Chains, Investment
Protection and Dispute Settlement Mechanisms''.
\url{http://web.mit.edu/insong/www/pdf/conjoint.pdf}.

Malesky, Eddy and Layna Mosley (2016).
\emph{Chains of Love? Global Production, Developing Country Firms and the Diffusion of Labor Standards}.
Working Paper. Niehaus Center.
\url{https://ncgg.princeton.edu/IPES/2016/papers/F1130_rm1.pdf}.

Osgood, Iain, Dustin Tingley, Thomas Bernauer, In Song Kim, Helen V.
Milner and Gabriel Spilker (2017). ``The Charmed Life of Superstar
Exporters: Survey Evidence on Firms and Trade Policy''. In:
\emph{Journal of Politics} 79.1, pp.~133-152.

Pandya, Sonal S. (2010). ``Labor Markets and the Demand for Foreign
Direct Investment''. In: \emph{International Organization} 64.3,
pp.~389-409.

Graduate students:

Imai, Kosuke, In Song Kim and Steven Liao (2017). ``Measuring Trade
Profiles with Two Billion Observations of Product Trade''.
\url{https://imai.princeton.edu/research/files/BIGtrade.pdf}.

\subsection{Week 11, 11/05 - 11/09: Political Economy of Corruption I:
Bureaucracy}\label{week-11-1105---1109-political-economy-of-corruption-i-bureaucracy}

\subsection{Week 12, 11/12 - 11/16: Political Economy of Corruption II:
Elections \&
Parties}\label{week-12-1112---1116-political-economy-of-corruption-ii-elections-parties}

\subsection{Week 13, 11/19 - 11/23: Political Economy of Corruption III:
Firms \&
Politics}\label{week-13-1119---1123-political-economy-of-corruption-iii-firms-politics}

\subsection{Week 14, 11/26 - 11/30: Putting It Together: States,
Development \&
Power}\label{week-14-1126---1130-putting-it-together-states-development-power}

Deep Dive:

Tilly, Charles (1992).
\emph{Coercion, Capital and European States, AD 990-1992}. Cambridge,
MA: Blackwell.

For graduate students:

Besley, Timothy (2007).
\emph{Principled Agents?: The Political Economy of Good Government}.
Oxford University Press.

\subsection{\texorpdfstring{Week 15, 12/03 - 12/07 \emph{Final Papers
Due}}{Week 15, 12/03 - 12/07 Final Papers Due}}\label{week-15-1203---1207-final-papers-due}




\end{document}

\makeatletter
\def\@maketitle{%
  \newpage
%  \null
%  \vskip 2em%
%  \begin{center}%
  \let \footnote \thanks
    {\fontsize{18}{20}\selectfont\raggedright  \setlength{\parindent}{0pt} \@title \par}%
}
%\fi
\makeatother
